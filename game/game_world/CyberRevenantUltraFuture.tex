\documentclass{report}
\usepackage[utf8]{inputenc}
\usepackage{amsmath}
\usepackage{listings}
\usepackage{hyperref}
\usepackage{graphicx}
\title{Sections and Chapters}
\author{Anton Nørgaard}
\date{ \today}
  
\begin{document}
  
\maketitle
  
\tableofcontents

\section{Introduction}
\chapter{Prerequisites for compiling and running the game}   
This chapter details all the necessary requirements for setting up, preparing and running the game. 

It wort noting that the game can \textbf{only} run on Linux operating systems as it uses certain C-libraries that only work in Unix-like operating systems.

Furthermore, at the risk of sounding arrogant and snarky, this guide assumes you are quite competent with programming, Linux and computers in general. 
\section*{Getting the correct compiler}
The game was and should be compiled using GCC. To my recollection, it uses GCC-specific, non c-standard features, so it is unlikely it will work otherwise.

All of this of course, assumes you have sufficient privileges to install software on the desired host.
\subsection*{Installing GCC via Ubuntu}
Firstly, run
\begin{lstlisting}
sudo apt update
\end{lstlisting}
To update the packages list. Next, run

\begin{lstlisting}
sudo apt install build-essential
\end{lstlisting}
This actually installs a group of different tools, most of which you will need anyway to compile the project, like make as well.

To verify GCC has been installed, run

\begin{lstlisting}
gcc --version
\end{lstlisting}

This should print out the version of GCC in use.
\section*{Additional required software}
\subsection*{Installing ncurses}
The code uses the ncurses library to draw all the characters on the screen. It is therefore necessary in order to compile the code.
\subsubsection*{Installing ncurses on systems with apt}
You need to install two packages
\begin{enumerate}
\item  libncurses5-dev (developer’s libraries for ncurses)
\item  libncursesw5-dev (developer’s libraries for ncursesw)
\end{enumerate}
To install both, run
\begin{lstlisting}
sudo apt-get install libncurses5-dev libncursesw5-dev
\end{lstlisting}
\subsection*{Installing SQLite}
Revenant also uses SQLite to manage game state. Furthermore, it needs several other components for allowing compilation itself. There are actually several valid approaches for integrating, compiling and  running SQLite, but this particular setup is based on what at the time seemed the best.

\subsubsection*{Getting the SQLite code}
To get the SQLite source code, go to \url{https://www.sqlite.org/download.html} and download whatever is specified to be "C source code as an amalgamation, version xx.yy.zz.", like in the picture below

\includegraphics[width=\textwidth]{sqlite.png}
Alternatively, assuming you have wget and you know the version number(in the case 3.44.2), you could also run
\begin{lstlisting}
wget https://www.sqlite.org/2023/sqlite-amalgamation-3440200.zip
\end{lstlisting}

Then unzip the downloaded file.
\subsubsection*{Compiling the CLI interface}
The game uses the command-line interface to generate the game world and the like. To enable the command line, go into the directory
\begin{lstlisting}
gcc shell.c sqlite3.c -lpthread -ldl -lm -o sqlite3
\end{lstlisting}
This 
\section*{Getting,compiling and generating the actual game}
Now that everything has been set up, the time has come to actually get, compile and generate the game world. To do this, simply go in to the revenant/misc folder and run the following shell script
\begin{lstlisting}
./setup_gamle_world.sh
\end{lstlisting}
and then promptly sit back and relax. Chances are, this is going to take a while
\chapter{Source code documentation}
\section{Management of game state and status}
\subsection{Keeping track of the creatures and the player}

\subsection{Checking for active effects of player}
\subsection{Creatures}
In the following chapers, we will use several variables to compute various information regarding creatures in-game, their status and the way
that they interact with the environment and the player. For that purpose, we will use the following defintions, some are explicitly computed/stored in-game, others are just definitions and are not explicitly declared in-game. The definitions we will use are
\begin{itemize}
\item $min\_local$ the smallest local coordinate value possible 
\item $max\_local$ the biggest local coordinate value possible 
\item $Player_{global  \:coordinate}$  to represent a player's global coordinate, either X or Y coordinate 
\item $Creature_{global \: coordinate}$ to represent a creature's global coordinate, either X or Y coordinate 
\item The exact same variables are defined for the local coordinates 
\item $max\_moves$ is the maximum number of tiles a player/creature can move, before they move our of bounds of the current screen view. It is computed as  ($max\_local$ - $max\_local$) +1 
\item $Player_{set \: to \: min}$  is a subtraction of the players global coordinate s.t its local coordinate is the minimal. It is computed as  $Player_{global \: coordinate}$ - ($Player_{local \: coordinate}$ - $min\_local$)
\item $Player_{set \: to \: max}$  is an addition of the players global coordinate s.t its local coordinate is the maximal. It is computed as  $Player_{global \: coordinate}$ + ($max\_local$ - $Player_{local \: coordinate}$)
\end{itemize}
With that in mind, we can now go over the different mechanics that we need to consider for the creatures in game, such as spawning, movement, death, behavior, actions and so on

\subsubsection{Spawning creatures}
When it comes to spawning creatures, there are several things that we need to compute in relation to the player. One of them is the creatures local coordinates. The kicker about this is that while a player's local coordinates can be arbitrary without any hassle, the coordinates of the creature must be aligned relative to the players to that the view the creature has on screen matches that of the global coordinates. To do this, there are two cases that need to be considered, the trivial case and the nontrivial case. \newline

In the trivial case, the distance between the player and the creature is s.t the distance for the particular axis, does not exceed the boundaries of the screen \newline

In the non-trivial case, what we effectively do is reduce it to the nontrivial case. The way that we do this is by considering whether the axis-wise coordinate for the player is greater or smaller than the creature's just like in the non-trivial case. In the case that

\[ Player_{global  \:coordinate} > Creature_{global \: coordinate} \]

Then set the creature's local coordinate to
\[ Creature_{local \: coordinate} = max\_local - (Player_{set \: to \: max} - Creature_{global \: coordinate} ) \mod  max\_moves  \]


In the other case where 
\[ Player_{global  \:coordinate} < Creature_{global \: coordinate} \]

Instead, set the creature's local coordinate to

\[ Creature_{local \: coordinate} = min\_local + (Player_{set \: to \: min} - Creature_{global \: coordinate} ) \mod  max\_moves  \]

The workings of this are a little hard to work around. Admittedly, this took rather a long while to figure out and I am unsure whether the computations can get simplified further. However, truth be told, I spent way too long to figure this out and so at this point that I cannot be bothered to come up with more clever solutions. \newline

The way it works (informally) is as follows. Assume that



\section{Controls}
\subsection{Keyboard Layout}
\begin{table}[h!]
\centering
\begin{tabular}{|c| c| c |c|} 
\hline
 a-m & n-z & A-M & N-Z+ESC \\ [0.5ex] 
 \hline
 a - N/A & n - N/A & A - N/A & N - N/A \\ 
 b - N/A & o - N/A & B - N/A & O - N/A \\
 c - N/A & p - N/A & C - N/A & P - N/A  \\
 d - N/A & q - N/A & D - N/A & Q - N/A  \\
 e - equip item in inventory & r - N/A & E - Show equipped items & R - N/A \\ 
 f - N/A & s - N/A & F - N/A & S - N/A \\
 g - N/A & t - N/A & G - N/A & T - N/A \\
 h - N/A & u - N/A & H - N/A & U - N/A \\
 i - Display current inventory & v - N/A & I - N/A & W - N/A \\
 j - N/A & w - N/A & J - N/A & X- N/A \\
 k - N/A & x - N/A & K - N/A & Y - N/A \\
 l - show past 10 events & y - N/A & L - N/A & Z - N/A \\
 m - N/A & z - N/A & M - N/A & ESC - Quit command \\ [1ex] 
 \hline
\end{tabular}
\caption{Keyboard bindings.}
\label{table:1}
\end{table}




\subsection{Character meanings on map}
\begin{table}[h!]

\begin{tabular}{|p{0.2\linewidth}|p{0.2\linewidth}| p{0.2\linewidth}|p{0.2\linewidth}|p{0.2\linewidth}|} 
\hline
 a-m & n-z & A-M & N-Z+ESC & Misc. \\ [0.5ex] 
 \hline
 a - animal & n - N/A & A - N/A & N - N/A & @ - player character\\ 
 b - N/A & o - N/A & B - N/A & O - N/A  & ! - interactable npc \\ 
 c - N/A & p - N/A & C - N/A & P - N/A \\
 d - N/A & q - N/A & D - N/A & Q - N/A  \\
 e - N/A & r - N/A & E - N/A & R - N/A \\
 f - N/A & s - N/A & F - N/A & S - N/A \\
 g - N/A & t - trader & G - N/A & T - N/A \\
 h - N/A & u - N/A & H - N/A & U - N/A \\
 i - N/A & v - N/A & I - N/A & W - N/A \\
 j - N/A & w - N/A & J - N/A & X- N/A \\
 k - N/A & x - N/A & K - N/A & Y - N/A \\
 l - N/A & y - N/A & L - N/A & Z - N/A \\
 m - N/A & z - N/A & M - N/A & ESC - Quit command \\ [1ex] 
 \hline
\end{tabular}
\caption{Meaning of characters on screen.}
\label{table:1}
\end{table}

\chapter*{The world of Revenant}

\section*{Places of Revenant}
\subsection*{Settlements}
\subsubsection*{Iislog}
An incredibly modest 
\section*{Who you are}





















\end{document}
